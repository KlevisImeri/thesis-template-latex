%----------------------------------------------------------------------------
\chapter{Introduction}
%----------------------------------------------------------------------------

Software has become indispensable in modern life, permeating everything from smartphones to 
satellites. Software inevitably contains bugs that require fixing. Software verification is 
the process of identifying these bugs in computer programs or demonstrating their absence. ~\cite{witness2} 
Various software verification methods exist, ranging from human-driven approaches like code 
reviews and software testing (including unit and integration testing) to automated techniques 
such as static and dynamic analysis. Formal methods, including model checking, offer mathematical 
approaches to software verification.

Numerous verification tools have been created by different organizations and research teams. 
Some prominent examples are \texttt{CPAchecker}, \texttt{UAutomizer}, \texttt{Theta}.
Despite being developed independently, these tools can still share their results with other tools
through witnesses.

These witnesses encapsulate verification information, enabling tools to validate results quicker,
either by themselves or by other verification tools. The witness format undergoes
continuous updates, with each iteration aiming to improve upon its predecessor. However, as
these witness formats evolve, tools must also integrate these new versions into their frameworks.
\texttt{Theta}, despite being a quite feature-rich verification framework, previously lacked a
validator for the Witness 2.0 version. This paper endeavors to lay the groundwork by outlining
the algorithms and implementation strategy for creating a validator for this witness type
within \texttt{Theta}. The proposed approach for implementing violation witness validation for
this new format involves converting the witness into an XCFA and subsequently
creating a product automaton with the program XCFA. This enables \texttt{Theta}'s
supported \texttt{multianalysis} to be run on this product automaton. Although the implemented
validator currently lacks several features, initial results have demonstrated promising outcomes.
The impact of this work lies in establishing the foundation for a feature-rich,
state-of-the-art validator.

